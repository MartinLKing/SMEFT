%!TEX root=main.tex
\section{Envoi}

Our analysis of how three philosophical accounts of model address bottom-up EFT strategies, in particular SM-EFT, raises several points of interest along the way.
To begin with, the analysis of SM-EFT puts Hartmann's main question, the relationship between model and theory in assessing EFT, back into the focus and gives it a somewhat surprising twist.
SM-EFT at Stage 1 can fail to be a model, at least on the MaM account, and yet still qualify as a theory, while it may have acquired, at Stage 3, the characteristics of a model.
SM-EFT is a theoretical tool that helps experimental physicists to parametrize their constraints or observations of deviations from the SM. 
Such searches may or may not be guided by any explicit models or aspects of models ([reference omitted]).

A bottom-up EFT is certainly not guided by a model or a theory in the usual sense. But our case study clearly shows that there exists a role for theory in scientific practice without the mediation of models. 
This is somewhat at odds both with the general idea of models as mediators between theory and data, and the hierarchical approach of Hartmann and advocates of the semantic approach. 
To be sure, theory in this case does not predict, unify, or explain. 
It is primarily a tool for the parametrization of experimental results that involve research motives that are not connected to specific models or theories.


Another philosophical result of our case study is that whether something is a model or not, may change within the different stages of an epistemic strategy and also depend on the status of empirical knowledge available or assumed.
Even though we have described the SM-EFT strategy as going through different stages, they need not occur in real time or in sequence. For, the SM-EFT framework can be used at the same time in one sector of measurements as a tool to describe constraints on deviations from the SM, and in a different sector of measurements as a Stage~3 EFT model of observed deviations from the SM.
It is also possible that a model that results as Stage 3 of a successful SM-EFT strategy in a bottom-up approach could also have been proposed independently in a top-down approach on the basis of some physics idea. Our point is that modelhood can change within the context of an epistemic strategy.


We have here examined one of the most interesting philosophical aspects of a recent change in particle physics characterized by a turn towards model independence. 
We have attempted to understand how and to what degree this bottom-up SM-EFT approach can be understood as independent from models. 
We have found that understanding its relation to models and theories is helped by looking at both the semantic and practice-based accounts that focus on different aspects of what it means to be a model. 
A careful look at the SM-EFT approach revealed three distinct stages which each bear different relations to theory and to the world. 
The generic bottom-up SM-EFT lacks a target of representation and thus lacks a key characteristic of a model. Focussing the SM-EFT on specific experimental sectors in a pragmatic way makes the SM-EFT experimentally and phenomenologically useful, but does not add a target of representation. Additional theoretical input or experimental evidence of a deviation from the SM can precisify a target of representation and the SM-EFT gains autonomy from the SM and QFT. It is only in this last stage that it takes on the characteristics of a model.
We take this as a reason not to take a view on models that is overly permissive: it blurs the lines between the different stages of the use of SM-EFT and glosses over the methodological aims of undertaking the bottom-up approach in the first place. 
Or put otherwise, not every step in a research strategy that successfully combines theoretical tools with experimental investigations and eventually leads to a new scientific model should be considered as a model itself.









