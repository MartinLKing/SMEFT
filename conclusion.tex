%!TEX root=main.tex
\section{Conclusion}


In this paper have examined one of the most interesting philosophical aspects of a recent change in particle physics, the turn towards model-independent and experiment-focused research at the example of the bottom-up SM-EFT approach. 
We have found that understanding its relation to models and theories not only depends on the philosophical understanding of models 
but also on the stages of this research practice.The generic bottom-up SM-EFT lacks a target of representation and thus lacks a key characteristic of a model. Focussing the SM-EFT on specific experimental sectors in a pragmatic way makes the SM-EFT experimentally and phenomenologically useful, but does not add a sufficiently well-defined target. Additional theoretical input or experimental evidence of a deviation from the SM can precisify a target of representation and the SM-EFT gains autonomy from the SM. It is only in this last stage that it takes on the characteristics of a model.
We take this as a reason -- motivated not primarily by philosophical theory but by existing research practices -- not to take a view on models that is overly permissive: it blurs the lines between the different stages of the use of SM-EFT and glosses over particle physicists' methodological aims of undertaking the bottom-up approach in the first place. 
Or put otherwise, not every step in a research strategy that successfully combines theoretical tools with experimental investigations and eventually leads to a new scientific model should be considered as a model itself.

Another philosophical result of our case study is that whether something is a model or not, may change within the different stages of an epistemic strategy and also depend on the status of empirical knowledge available or assumed.
Even though we have described the SM-EFT strategy as going through different stages, they need not occur in real time or in sequence. For, the SM-EFT framework can be used at the same time in one sector of measurements as a tool to describe constraints on deviations from the SM, and in a different sector of measurements as a Stage~3 EFT model of observed deviations from the SM.
It is also possible that a model that results as Stage 3 of a successful SM-EFT strategy in a bottom-up approach could also have been proposed independently in a top-down approach on the basis of some physics idea. But as shown it is also possible that empirical evidence in terms of deviations from the SM prompts physicists to move through the three stages. Our point was just that modelhood can change within the context of an epistemic strategy without the prior application of specific modelling assumptions but through empirical evidence gained in its wake.







